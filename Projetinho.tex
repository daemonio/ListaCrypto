\documentclass{article}
\usepackage{graphicx}
\usepackage{url}

\usepackage[brazilian]{babel}
\usepackage[utf8]{inputenc}
\usepackage[T1]{fontenc}

\begin{document}

\title{%
  Compartilhamento de Internet Wi-fi com pagamento em IOTA \\
  \large Blockchains, Criptomoedas e Outras Aplicações}
    
\author{Marcos Paulo Ferreira}

\maketitle

\begin{abstract}
Nesse trabalho desenvolveremos um sistema de conexão Wi-fi em que um dispositivo primeiro realiza
o pagamento para a utilização através da moeda IOTA. Dessa maneira, um celular ao detectar uma rede
Wi-fi disponível terá que pagar uma quantidade de moeda para o roteador para se conectar ao ponto de acesso.
\end{abstract}


\section{Introdução}\label{sec:Introduction}

pass

%\cite{Homer_2007}, or perhaps at the origin of the famous "lorem ipsum"
%\cite{LoremIpsum_wiki}.

\section{Objetivos}\label{sec:Goals}

O objetivo é mostrar a eficiência e flexibilidade da moeda IOTA como meio de pagamento de serviços em uma economia
machine-to-machine. Será desenvolvido um pequeno aplicativo web que permite que um dispositivo cliente, como um
celular, se comunique com um roteador fornecedor de Internet por rede sem-fio. Inicialmente será proposto
um simples esquema de comunicação por interface Web, mas a ideia do trabalho pode ser generalizada para implementações
até mesmo em hardware.
Forneceremos como funcionará a comunicação, os trâmites das transações e como o roteador gerencia quais clientes
terão acesso à rede e por quanto tempo. Também elaboraremos um simples método para se calcular o custo total
de uso da rede que se baseia na capacidade total da rede no momento e de suas características, como largura de banda e
número de usuários ativos.

\section{Metodologia}\label{sec:Metodology}

pass

\section{Resultados}\label{sec:Output}

pass

\section{Conclusão}
pass


%\bibliographystyle{unsrt}
%\bibliography{my_references}
\end{document}
