\documentclass{article}
\usepackage{graphicx}
\usepackage{url}

\usepackage[brazilian]{babel}
\usepackage[utf8]{inputenc}
\usepackage[T1]{fontenc}

\begin{document}

\title{%
  Compartilhamento de Internet Wifi com pagamento em IOTA \\
  \large Blockchains, Criptomoedas e Outras Aplicações}
    
\author{Marcos Paulo Ferreira}

\maketitle

\begin{abstract}
Nesse trabalho desenvolveremos um sistema de conexão Wifi em que um dispositivo primeiro realiza
o pagamento para a utilização através da moeda IOTA. Dessa maneira, um celular ao detectar uma rede
Wifi disponível terá que pagar uma quantidade de moeda para o roteador para se conectar ao ponto de acesso.
\end{abstract}


\section{Introdução}\label{sec:Introduction}

pass

%\cite{Homer_2007}, or perhaps at the origin of the famous "lorem ipsum"
%\cite{LoremIpsum_wiki}.

\section{Objetivos}\label{sec:Goals}

O objetivo é mostrar a eficiência e flexibilidade da moeda IOTA como meio de pagamento de serviços em uma economia
machine-to-machine. Será desenvolvido um pequeno aplicativo web que permite que um dispositivo cliente, como um
celular, obtenha o endereço IOTA do roteador fornecedor de Internet sem-fio. O sistema é parecido com um weblogin
muito comum em hospitisl e cafés, porém ao invés de obter acesso por login/senha, o cliente só será aceito na rede
após a realização do pagamento de um valor estipulado.
Forneceremos como funcionará a comunicação, os trâmites das transações e como o roteador gerencia quais clientes
terão acesso à rede e por quanto tempo. Também elaboraremos um simples método para se calcular o custo total
de uso da rede que se baseia na capacidade total da rede no momento e de suas características, como largura de banda e
número de usuários.

\section{Metodologia}\label{sec:Metodology}

\subsection{Visão Geral}

Quando um cliente descobre que há uma rede IOTA Wifi disponível ele se conecta a ela. A rede Wifi deve ser aberta para todos
para que, ao se conectar, o cliente adquira um endereço IP na mesma faixa do roteador e assim um possível canal de
comunicação seja formado. O cliente está conectado na rede mas seu acesso não está liberado -- se ele tentar abrir alguma
página Web, ele seja direcionado para a página de informações sobre o pagamento em IOTA, e nessa página, o cliente recebe
o endereço IOTA do roteador e será avisado que o acesso à Internet só será liberado após o pagamento da quantia especificada.

O cliente procede com a transação utilizando a sua carteira IOTA e quase instantaneamente o roteador reconhece o pagamento e,
sem seguida, libera o endereço MAC do cliente para acesso.

\subsection{Configuração do roteador}

O roteador possui sua carteira IOTA, como também scripts para aceitar e rejeitar novos clientes. O roteador é livre para usar
somente um endereço IOTA fixo ou variável, pois esse endereço é mostrado na página Web inicial. A liberação do cliente é feita ao
inserir o endereço MAC do cliente, obtido na mensagem da comunicação, em um arquivo de whitelist. Todo e qualquer cliente, inicialmente,
está em blacklist, e somente clientes que já pagaram o acesso estão em whitelist.

\subsection{Configuração do cliente}

\section{Resultados}\label{sec:Output}

pass

\section{Futuras melhorias}\label{sec:Future}

Uma lista de futuras melhorias para o sistema.

\begin{enumerate}
\item Retirar a página Web inicial e pesquisar por métodos de comunicação pré-conexão.
Seria necessário trabalhar em um novo protocolo de comunicação.
\item O roteador estipular um tempo máximo de conexão para o cliente ou estipular ``planos'' envolvendo preços
por tempo de uso.
\item O cliente anunciar para outros aparelhos -- em troca de IOTAs -- que há uma rede de bom custo benefício
de disponível. Seria necessário trabalhar em um novo protocolo de comunicação, como no item 1.
\end{enumerate}

\section{Conclusão}
pass


%\bibliographystyle{unsrt}
%\bibliography{my_references}
\end{document}
