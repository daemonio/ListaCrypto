\documentclass[a4paper]{article}
\usepackage{graphicx}
\usepackage{url}
\usepackage{amssymb}
\usepackage{amstext}
\usepackage{amsmath}
\usepackage[brazilian]{babel}
\usepackage[utf8]{inputenc}
\usepackage[T1]{fontenc}

\newcommand{\Mod}[1]{\ (\mathrm{mod}\ #1)}

\usepackage{listings}
\usepackage{color}

\definecolor{dkgreen}{rgb}{0,0.6,0}
\definecolor{gray}{rgb}{0.5,0.5,0.5}
\definecolor{mauve}{rgb}{0.58,0,0.82}

\lstset{frame=tb,
  language=Python,
  aboveskip=3mm,
  belowskip=3mm,
  showstringspaces=false,
  columns=flexible,
  basicstyle={\small\ttfamily},
  numbers=none,
  numberstyle=\tiny\color{gray},
  keywordstyle=\color{blue},
  commentstyle=\color{dkgreen},
  stringstyle=\color{mauve},
  breaklines=true,
  breakatwhitespace=true,
  tabsize=3
}


\graphicspath{ {/home/marcos/} }

\begin{document}

\title{%
  Bitcoin \& Blockchain \\
  \large {Lista 1}
}
    
\author{Marcos Paulo Ferreira}

\maketitle

\section{Exercício 1}\label{sec:Ex1}

Primeiro definimos a cifra:

\begin{lstlisting}
ocwy=['ocwyikoooniwugpmxwktzdwgtssayjzwyemdlbnqaaavsuwdvbrflauplooubfgq',
    'hgcscmgzlatoedcsdeidpbhtmuovpiekifpimfnoamvlpqfxejsmxmpgkccaykwf',
    'zpyuavtelwhrhmwkbbvgtguvtefjlodfefkvpxsgrsorvgtajbsauhzrzalkwuow',
    'hgedefnswmrciwcpaaavogpdnfpktdbalsisurlnpsjyeatcuceesohhdarkhwot',
    'ikbroqrdfmzghgucebvgwcdqxgpbgqwlpbdaylooqdmuhbdqgmyweuik'];

text = ''.join(ocwy)  
\end{lstlisting}

Depois encontramos a chave através do método proposto pelo autor. Uma função de nome ``find\_key\_length()'' recebe o
``displacement`` e retorna o número de coincidências. O resultado foi:

\begin{lstlisting}
"displacement:  2 coincidences:  13
displacement:  3 coincidences:  11
displacement:  4 coincidences:  6
displacement:  5 coincidences:  15
displacement:  6 coincidences:  23
displacement:  7 coincidences:  8
displacement:  8 coincidences:  5
displacement:  9 coincidences:  12"
\end{lstlisting}

O maior valor de coincidências é o 23 para o ''displacement'' 6, assim supomos que o tamanho da chave seja: $n = 6$.

Agora iteramos 6 vezes para encontrar os 6 caracteres da chave. Uma função de nome ''get\_W()`` recebe o texto cifrado,
o tamanho do chave, o número da iteração e retorna o vetor $W$ de frequências resultante do i-deslocamento.

\begin{lstlisting}
for i in range(N):
    print 'Deslocando: ', i
    W = get_W(ALPHA, text, N, i)
    
    print W
    
"Deslocando:  0
[0.057, 0.076, 0.038, ...]
Deslocando:  1
[0.057, 0.134, 0.096, ...]
Deslocando:  2
[0.019, 0.0, 0.057, ...]
Deslocando:  3
[0.096, 0.0386, 0.0, ...]
Deslocando:  4
[0.038, 0.0, 0.0, ...]
Deslocando:  5
[0.096, 0.0, 0.0384, ...]
"
\end{lstlisting}

Computamos para cada vetor $W$ acima o produto escalar com o vetor $A$ deslocado 25 vezes, $W \cdot A_j$,
sendo o $j = 1, .., 25$. O vetor $A$ representa as frequências de letras do alfabeto inglês. Assim,
teremos 25 valores escalares em um vetor $R$, como mostrado abaixo. A posição do maior elemento de $R$,
calculado com $R.index(max(R))$, é o valor da chave naquela posição.

\begin{lstlisting}
# Iteracao 1
R = [0.042, 0.045, 0.037, 0.045, 0.029, 0.03, 0.042, 0.056, 0.038, 0.04, 0.035, 0.037, 0.032, 0.039, 0.041, 0.039, 0.041, 0.032, 0.035, 0.038, 0.048, 0.032, 0.045, 0.039, 0.033]
R.index(max(R))
7

# Iteracao 2
[0.036, 0.044, 0.04, 0.038, 0.036, 0.031, 0.027, 0.038, 0.044, 0.041, 0.047, 0.031, 0.03, 0.047, 0.064, 0.034, 0.026, 0.038, 0.042, 0.036, 0.036, 0.037, 0.033, 0.043, 0.044]
R.index(max(R))
14

# Iteracao 3
[0.047, 0.038, 0.036, 0.041, 0.035, 0.029, 0.033, 0.041, 0.033, 0.031, 0.045, 0.061, 0.039, 0.031, 0.04, 0.046, 0.033, 0.034, 0.041, 0.032, 0.027, 0.044, 0.048, 0.037, 0.036]
R.index(max(R))
11

# Iteracao 4
[0.046, 0.041, 0.039, 0.031, 0.039, 0.034, 0.034, 0.029, 0.048, 0.039, 0.036, 0.035, 0.061, 0.044, 0.029, 0.035, 0.044, 0.028, 0.031, 0.043, 0.035, 0.031, 0.041, 0.045, 0.041]
R.index(max(R))
12

# Iteracao 5
[0.051, 0.036, 0.032, 0.035, 0.06, 0.039, 0.033, 0.038, 0.045, 0.027, 0.036, 0.044, 0.036, 0.036, 0.038, 0.045, 0.036, 0.042, 0.038, 0.043, 0.039, 0.036, 0.037, 0.034, 0.035]
R.index(max(R))
4

# Iteracao 6
[0.031, 0.033, 0.052, 0.047, 0.026, 0.033, 0.049, 0.039, 0.031, 0.032, 0.04, 0.03, 0.038, 0.036, 0.04, 0.039, 0.035, 0.037, 0.064, 0.039, 0.027, 0.035, 0.049, 0.029, 0.042]
R.index(max(R))
18

KEY = [7, 14, 11, 12, 4, 18]
\end{lstlisting}

Assim, temos a chave KEY $= [7, 14, 11, 12, 4, 18]$. Usando uma função de decriptar, encontramos a mensagem original:

\begin{lstlisting}
print decrypt(text, KEY)

"holmeshadbeenseatedforsomehoursinsilencewithhislongthinbackcurvedovera\
chemicalvesselinwhichhewasbrewingaparticularlymalodorousproducthishead\
wassunkuponhisbreastandhelookedfrommypointofviewlikeastrangelankbird\
withdullgreyplumageandablacktopknotsowatsonsaidhesuddenlyyoudonotpropose\
toinvestinsouthafricansecurities"
\end{lstlisting}

\section{Exercício 2}\label{sec:Ex2}

O atacante pode utilizar de padrões na mensagem original para descobrir a chave. Um campo de ``header'', por exemplo, pode ser o mesmo para várias mensagens, o que gerará a
mesma saída na cifra. Se o atacante conhece o conteúdo do header, para descobrir a chave ele simplesmente faz:

Como;

$c = m \oplus k$

então:

$k = c \oplus m$

\section{Exercício 3}\label{sec:Ex3}

Primeiro mostraremos porque OTP é segura:

Imagine uma mensagem de um bit. Assim, teremos uma chave de um bit e uma cifra de um bit. Dado o bit de entrada, o OTP garante que o bit de saída pode ser tanto $0$ ou $1$ na mesma
probabilidade, ou seja, os bits são distribuídos uniformemente. Com isso, o atacante não tem o poder de dizer se o bit $0$ tem maior probabilidade de sair que o bit $1$,
por exemplo. Logo, o sistema é probabilisticamente seguro -- qualquer bit possui a mesma probabilidade de sair -- e portanto, seguro se a chave for usada somente uma vez.

Devido a isso, um ataque de brute force necessariamente deve ser por via exaustiva porque o sistema não possui “bias” sobre a saída de bits. Como toda chave tem a mesma probabilidade
de ser a chave real, então o atacante terá que gerar todas as chaves possíveis e testar.

O sistema é seguro contra ataque brute force contanto que o tamanho da chave seja grande.

\section{Exercício 4}\label{sec:Ex4}

\subsection{Parte 1}

A chave é composta de 8 caracteres de 8 bits. Então: $64 = 8 * 8$ bits. Após o PC-1, cada caracter terá 7 bits, logo: $56 = 8 * 7$. A busca exaustiva terá que percorrer $2^{56}$ chaves.

Conseguimos gerar $10^6$ chaves por segundo, logo o tempo total será em média:

\framebox{$T\ =\ \frac{2^{56}}{2 * 10^6} = 36.028.797.018\ segundos$}

O resultado corresponde a: $1.141$ anos. A divisão por $2$ é para calcular o tempo médio.

\subsection{Parte 2}

Se a chave for composta de 8 caracteres de 7 bits, após o PC-1 o DES retirará um bit, então as chaves terão caracteres de 6 bits: $8 * 6 = 48$. A busca exaustiva demorará em média:

\framebox{$T\ = \frac{2^{48}}{2 * 10^6} = 140.737.488\ segundos$}

O resultado corresponde a: $4,46$ anos.

\subsection{Parte 3}

Temos 26 letras disponíveis, o total de chaves seria: $8 * 26 = 208$ chaves.

O número de bits para gerar essas 208 chaves seria:

\framebox{$\log_2 208 \approx 7,7\ bits$}.

A busca exaustiva é somente sobre $7,7$ bits, o total de tempo seria:

\framebox{$T\ =\ \frac{208}{2 * 10^6} \approx 0.000104\ segundos$}

Isso é, a busca exaustiva demoraria frações de segundos.

\section{Exercício 5}\label{sec:Ex5}

\subsection{Parte 1}

Os ASIC's calculam:

\framebox{$V = 10^5 * 3 * 10^7 \approx 3 * 10^{12}\text{ chaves/segundo}$}

O tamanho do domínio das chaves é: $2^{191}$, então o tempo médio para encontrar a chave será:

\framebox{$\frac{2^{191}}{2 * 3 * 10^{12}} \approx 10^{45}\text{ segundos}$}

Esse valor em anos é aproximadamente: $3 * 10^{37}\text{ anos}$. Comparando com a idade do universo, esse
valor é $10^{28}$ maior que a idade do universo.

\subsection{Parte 2}

Devemos dividir o total de anos encontrados por $2^i$ e igualar o resultado a $24$ horas ou $1$ dia. O valor de $i$ foi
o quanto o poder computacional dobrou de acordo com a Lei de Moore para que a computação pudesse ser realizada em um único dia.

$\frac{3 * 10^{37} * 365}{2^i} = 1\text{ dia}$

O valor de $i$ é o logaritmo base $2$ do valor a esquerda.

\framebox{$i = \log_2 {3*10^{37} * 365} \approx 133$}

O número de iterações é de $133$ que corresponde a:

\framebox{$133 * 1.5 \approx 200\text{ anos}$}
\newline
\newline
Considerando que a Lei de Moore se aplica a cada $1.5$ anos.

\section{Exercício 6}\label{sec:Ex6}

\subsection{Eletronic CodeBook Mode (ECB)}

A maneira mais simples de encriptação. A mensagem m é dividida em blocos e cada bloco $x_i$ é entrada para uma
cifra de bloco (ex: AES) junto com a chave privada k. A saída é a concatenação de cada cifra $y_i$.

Vantagens:

\begin{enumerate}
\item Simples de programar e Paralelizável.
\item Não precisa de sincronização de blocos para decriptar.
\item Erro em um bloco prejudica somente aquele bloco.
\end{enumerate}

Desvantagens:

\begin{enumerate}
\item Processo determinístico. Um bloco $x_i$, para a mesma chave k, gera uma mesma cifra $y_i$. Atacante pode
utilizar desse fato para realizar o ataque de análise de tráfego.
\item Atacante pode reordenar os blocos das cifras e ainda sim obter uma cifra válida (ataque de substituição).
\item Cifra pode ser alterada e ser considerada válida (cifra sem integridade).
\end{enumerate}

\subsection{Cipherblock Channing Mode (CBC)}

Semelhante ao CBC, porém as cifras são encadeadas de modo que uma cifra $y_i$ dependa tanto de $x_i$ quanto da
cifra passada, $y_{i-1}$. Isso cria um processo de encadeamento com a propriedade de que cada cifra dependa
das cifras anteriores. Para a primeira cifra, já que não existe cifra anterior a ela, é usado um número inicial,
chamado de $IV$.

Vantagens:

\begin{enumerate}
\item Devido ao encadeamento de cifras, o CBC resolve o problema do determinismo, já que agora um mesmo bloco
de mensagem não gera a mesma cifra, como acontecia no ECB.
\item Ainda devido ao encadeamento, é moderadamente resistente ao ataque de substituição. (Ver desvantagens).
\item Se o IV é novo a cada processo, então CBC vira um esquema de criptografia probabilística. Dessa maneira,
o sistema é resistente ao ataque de análise de tráfego.
\item IV não precisa ser segredo, mas é importante que seja uma nonce (usado somente uma vez).
\end{enumerate}

Desvantagens:

\begin{enumerate}
\item Não garante integridade da mensagem, logo pode ser atacado pela modificação da cifra.
\item Dependendo da semântica da cifra (exemplo: transferência bancária), um atacante pode substituir algum bloco
por valor aleatório, fazendo com que a transferência bancária seja para algum lugar desconhecido, e com valor desconhecido, e
isso obviamente é algo ruim.
\end{enumerate}

\subsection{Counter Mode (CTR)}

A chave de ciframento é calculada usando o algoritmo de cifra de bloco ex: AES) e depois usada de modo semelhante como nas stream ciphers. A entrada
do AES é um contador, incrementado em cada bloco, e a saída é a chave usada na operação xor com o respectivo bloco da mensagem.

Vantagens:

\begin{enumerate}
\item Como cada cifra é calculada de forma independente, pois a entrada é o valor do contador, então o esquema como um todo
pode ser paralelizável.
\item Devido ao contador, blocos iguais terão contadores diferentes, logo cifras diferentes. O esquema é resistente ao ataque
de análise de tráfego.
\end{enumerate}

Desvantagens:

\begin{enumerate}
\item Necessário que o contador inclua também um IV. Caso contrário, o mesmo valor de contador pode ser usado várias vezes, o
que facilita para o atacante conhecer a chave da cifra de bloco.
\item Não garante integridade das cifras, logo pode sofrer tempering.
\end{enumerate}

\subsection{Galois Counter Mode (GCM)}

É um modo de encriptação que mantém as características do CTR: o uso de contadores e possibilidade de operar em paralelo.

A novidade é que o GCM calcula o MAC (Message Authentication Code) da cifra. O MAC é usado para verificar se a cifra sofreu alterações ou
não, o que garante a integridade da mesma. Desse modo, qualquer ataque sobre a cifra (ex: ataque de substituição e tempering) não irá
funcionar. Quando a cifra chega no outro lado, o destino recalcula o MAC e verifica se ele é igual ao recebido, se sim, ele considera
a cifra como válida e segue o processo de decriptação. Se não, ele descarta a cifra recebida.

Vantagens:

\begin{enumerate}
\item Mantém as características da CTR: cifras não determinísticas e paralelização.
\item Garante integridade das cifras. Impossibilita ataques de substituição e tempering.
\item Posssibilita o uso de um campo adicional, o ADD, que pode ser usada para autenticação da mensagem. Desse modo, Alice pode provar
que foi ela quem enviou a mensagem para Bob.
\item O cálculo do MAC é relativamente rápido, logo não contribui em overhead.
\end{enumerate}

Desvantagens:

\begin{enumerate}
\item Aumenta o tamanho da cifra devido ao acréscimo do MAC e também, possivelmente, do ADD.
\end{enumerate}

\section{Exercício 7}\label{sec:Ex7}
$\vert\mathbb{Z}_{53}^{*}\vert = 52 = 13 * 2 * 2$

As possívels ordens dos elementos devem dividir $52$, logo elas são: $1$, $2$, $4$, $13$, $26$.

Usando $5$ como gerador (ou elemento primário) e de posse de uma calculadora
de mesa, os seguintes subgrupos possuem:

\begin{enumerate}
\item subgrupo de ordem $1$ possui $1$ elementos
\item subgrupo de ordem $13$ possui $12$ elementos
\item subgrupo de ordem $2$ possui $1$ elementos
\item subgrupo de ordem $26$ possui $12$ elementos
\item subgrupo de ordem $4$ possui $2$ elementos
\item subgrupo de ordem $52$ possui $25$ elementos
\end{enumerate}

\section{Exercício 8}\label{sec:Ex8}

\subsection{Parte 1}

\begin{enumerate}
\item $\vert\mathbb{Z}_{5}^{*}\vert = 4$

\item $\vert\mathbb{Z}_{7}^{*}\vert = 6$

\item $\vert\mathbb{Z}_{13}^{*}\vert = 12$
\end{enumerate}

\subsection{Parte 2}

Sim. Devido ao Teorema apresentado.

\subsection{Parte 3}

Usando uma calculadora de mesa, obtemos:

\begin{enumerate}
\item $\vert\mathbb{Z}_{5}^{*}\vert: 2, 3$

\item $\vert\mathbb{Z}_{7}^{*}\vert: 3, 5$

\item $\vert\mathbb{Z}_{13}^{*}\vert:2, 6, 7, 11$
\end{enumerate}

\subsection{Parte 4}


\begin{enumerate}
\item $\phi(\vert\mathbb{Z}_{5}^{*}\vert) = \phi(4) = 2$

\item $\phi(\vert\mathbb{Z}_{7}^{*}\vert) = \phi(6) = 2$

\item $\phi(\vert\mathbb{Z}_{13}^{*}\vert) = \phi(12) = 12$
\end{enumerate}


\section{Exercício 9}\label{sec:Ex9}

\subsection{Parte 1}

\begin{enumerate}
\item $K_{pubA} = 2^3 \Mod {467} = 8$
\item $K_{pubB} = 2^5 \Mod {467} = 32$
\item $K_{AB} = (2^3)^5 \Mod {467} = 78$
\end{enumerate}

\subsection{Parte 2}

\begin{enumerate}
\item $K_{pubA} = 2^{400} \Mod {467} = 137$
\item $K_{pubB} = 2^{134} \Mod {467} = 84$
\item $K_{AB} = (2^{400})^{137} \Mod {467} = 90$
\end{enumerate}

\subsection{Parte 3}

\begin{enumerate}
\item $K_{pubA} = 2^{228} \Mod {467} = 394$
\item $K_{pubB} = 2^{57} \Mod {467} = 313$
\item $K_{AB} = (2^{228})^{57} \Mod {467} = 206$
\end{enumerate}

\section{Exercício 10}\label{sec:Ex10}

O valor $a = 1$ resulta em uma chave pública $K_A = \alpha^1 = \alpha$. Esse valor caminharia em um meio
inseguro e o atacante perceberia que a chave privada é $1$ devido a chave pública ser igual ao número público $\alpha$.

O valor $a = p-1$ resultaria em valor $1$ para a chave pública, devido ao Teorema de Fermat: $K_A = \alpha^{p-1} \equiv 1 \Mod p$. Se o
atacante ver no meio inseguro o valor $1$ para a chave pública, ele consegue facilmente deduzir o valor da chave privada.

\section{Exercício 11}\label{sec:Ex11}

\begin{lstlisting}
a = 106
b = 12375
p = 24691

k = (b ** ((p-1)/2)) % p

for i in range(p):
    if k == 1 and i%2 == 0:
        if (a ** i) % p == b:
            print a,"^",i,"=",b
            break
    if k == -1 and i%2 != 0:
        if (a ** i) % p == b:
            print a,"^",i,"=",b
            break
\end{lstlisting}

Testando para os valores dados, a saída foi:

$106^{22392} = 12375$. Assim: \framebox{$x = 22392$}

\section{Exercício 12}\label{sec:Ex12}

\subsection{Parte 1}
$(2, 7) + (5, 2)$ usando $y^2 \equiv x^3 + 2x + 2 \Mod{17}$. 
\newline
\newline
Temos: $a = 2, b = 2, x_1 = 2, y_1 = 7, x_2 = 5, y_2 = 2$
%$s = \frac{3x_1^2 + a}{2y_1} = (3*2^2 + 2) * (2*7)^{-1} = 14*14^{-1} \equiv 1 \Mod{17}$

\begin{equation}
  \label{eq:t}
  \begin{align*}
    %s &= \frac{3x_1^2 + a}{2y_1} = (3*2^2 + 2) * (2*7)^{-1} = 14*14^{-1} \equiv 1 \Mod{17}\\
    s &= \frac{y_2 - y_1}{x_2 - x_1} = (2 - 7)/(5 - 2) = -5 * 3^{-1} = 12 * 6 \equiv 4 \Mod{17}\\
    x_3 &= s^2 - x_1 - x_2 = 4^2 - 2 - 5 \equiv 9 \Mod{17}\\
    y_3 &= s(x_1 - x_3) - y_1 = 4(2 - 9) - 7 = -35 \equiv 16 \Mod{17}
  \end{align*}
\end{equation}

O resultado é: $(9, 16)$

Conferindo:

\begin{equation}
  \label{eq:t}
  \begin{align*}
    y^2 \equiv x^3 + 2x + 2 \Mod{17}\\
    16^2 \equiv 9^3 + 2*9 + 2 \\
    1 \equiv 1 \Mod{17}
  \end{align*}
\end{equation}

\subsection{Parte 2}
$(3, 6) + (3, 6)$ usando $y^2 \equiv x^3 + 2x + 2 \Mod{17}$. 
\newline
\newline
Temos: $a = 2, b = 2, x_1 = x_2 = 3, y_1 = y_2 = 6$
%$s = \frac{3x_1^2 + a}{2y_1} = (3*2^2 + 2) * (2*7)^{-1} = 14*14^{-1} \equiv 1 \Mod{17}$

\begin{equation}
  %\label{eq:t1}
  \begin{align*}
    s &= \frac{3x_1^2 + a}{2y_1} = (3*3^2 + 2) * (2*6)^{-1} = 29*12^{-1} = 12 * 12^{-1} \equiv 1 \Mod{17}\\
    x_3 &= s^2 - x_1 - x_2 \equiv 1 \Mod{17}\\
    y_3 &= s(x_1 - x_3) - y_1 = 1(3 - 1) -6 \equiv -4 \equiv 13 \Mod{17}
  \end{align*}
\end{equation}

O resultado é: $(1, 13)$

Conferindo:

\begin{equation}
  %\label{eq:t2}
  \begin{align*}
    y^2 \equiv x^3 + 2x + 2 \Mod{17}\\
    13^2 \equiv 1^3 + 2*1 + 2 \\
    16 \equiv 5 \Mod{17}
  \end{align*}
\end{equation}

Podemos ver que o ponto não está na curva.

\section{Exercício 13}\label{sec:Ex13}

\begin{equation}
  %\label{eq:t3}
  \begin{align*}
    p+1 - 2\sqrt{p} \leq \# E \leq p + 1 + 2\sqrt{p}\\
    18 - 2*4.12 \leq \# E \leq 18 + 2*4.12\\
    9.76 \leq \# E \leq 26.24
  \end{align*}
\end{equation}

Para $E = 19$ a inequação se mantém.

\section{Exercício 14}\label{sec:Ex14}

É porque cada ponto pode ser reescrito em função de P apenas multiplicando uma constante a ele. Devido a propriedade da adição,
cada ponto está distante um do outro por uma constante multiplicada a P.

Um ponto qualquer na curva tem o formato $Q = iP$, se somado consigo mesmo, tem-se o resultado: $2Q = (i+1)P$. O processo se repete
para $3Q, 4Q, ...$ até que todos os números tenham sido gerados.

\section{Exercício 15}\label{sec:Ex15}

\subsection{Parte 1}

\begin{equation}
  \begin{align*}
    \text{Começamos de P = }(0, 3)\\
    \text{ Queremos }P+P = 2P\\
    (0, 3) + (0, 3) = (x_3, y_3)\\
    s &= \frac{3x_1^2 + a}{2y_1} = (3*0^2 + 3) * (6)^{-1} = 3 * 6 \equiv 4 \Mod{7}\\
    x_3 &= s^2 - x_1 - x_2 = 16 \equiv 2 \Mod{7}\\
    y_3 &= s(x_1 - x_3) - y_1 = 2(0 - 4) - 3 \equiv -11 \equiv 3 \Mod{7}\\
    \text{O ponto é: }(2, 3)
  \end{align*}
\end{equation}

\begin{equation}
  \begin{align*}
    \text{ Queremos }2P+P = 3P\\
    (2, 3) + (0, 3) = (x_3, y_3)\\
    s &= \frac{y_2 - y_1}{x_2 - x_1} = (3 - 3)/(0 - 2) \equiv 0 \Mod{7}\\
    x_3 &= s^2 - x_1 - x_2 = -2 \equiv 5 \Mod{7}\\
    y_3 &= s(x_1 - x_3) - y_1 = -3 \equiv 4 \Mod{7}\\
    \text{O ponto é: }(5, 4)
  \end{align*}
\end{equation}

\begin{equation}
  \begin{align*}
    \text{ Queremos }3P+P = 4P\\
    (5, 4) + (0, 3) = (x_3, y_3)
    s &= \frac{y_2 - y_1}{x_2 - x_1} = (3 - 4)/(0 - 5) = 1 * 5^{-1} \equiv 3 \Mod{7}\\
    x_3 &= s^2 - x_1 - x_2 = 9 - 5 - 0 \equiv 4 \Mod{7}\\
    y_3 &= s(x_1 - x_3) - y_1 = 3(5 - 4) - 4 = -1 \equiv 6 \Mod{7}\\
    \text{O ponto é: }(4, 6)
  \end{align*}
\end{equation}

\begin{equation}
  \begin{align*}
    \text{ Queremos }4P+P = 5P\\
    (4, 6) + (0, 3) 
    s &= \frac{y_2 - y_1}{x_2 - x_1} = (3 - 6)/(0 - 4) = 3 * 4^{-1} \equiv 6 \Mod{7}\\
    x_3 &= s^2 - x_1 - x_2 = 36 - 4 - 0 = 32 \equiv 4 \Mod{7}\\
    y_3 &= s(x_1 - x_3) - y_1 = 6(4 - 4) - 6 = -6 \equiv 1 \Mod{7}\\
    \text{O ponto é: }(4, 1)
  \end{align*}
\end{equation}

\begin{equation}
  \begin{align*}
    \text{ Queremos }5P+P = 6P\\
    (4, 1) + (0, 3) = (x_3, y_2)\
    s &= \frac{y_2 - y_1}{x_2 - x_1} = (3 - 1)/(0 - 4) = -2 * 4^{-1} = -4 \equiv 3 \Mod{7}\\
    x_3 &= s^2 - x_1 - x_2 = 9 - 4 - 0 \equiv 5 \Mod{7}\\
    y_3 &= s(x_1 - x_3) - y_1 = 3(4 - 5) - 1 = -4 \equiv 3 \Mod{7}\\
    \text{O ponto é: }(5, 3)
  \end{align*}
\end{equation}

\begin{equation}
  \begin{align*}
    \text{ Queremos }3P+4P = 7P\\
    (5, 4) + (4, 6) = (x_3, y_3)\\
    s &= \frac{y_2 - y_1}{x_2 - x_1} = (6 - 4)/(4 - 5) = -2 \equiv 5\Mod{7}\\
    x_3 &= s^2 - x_1 - x_2 = 25 - 5 - 4 = 16 \equiv 2 \Mod{7}\\
    y_3 &= s(x_1 - x_3) - y_1 = 5(5 - 2) - 4 = 11 \equiv 4 \Mod{7}\\
    \text{O ponto é: }(2, 4)
  \end{align*}
\end{equation}

\begin{equation}
  \begin{align*}
    \text{ Queremos }7P+P = 8P\\
    (2, 4) + (0, 3) = (x_3, y_2)\
    s &= \frac{y_2 - y_1}{x_2 - x_1} = (3 - 4)/(0 - 2) = 1 * 2^{-1} \equiv 4 \Mod{7}\\
    x_3 &= s^2 - x_1 - x_2 = 16 - 2 - 0 = 14 \equiv 0 \Mod{7}\\
    y_3 &= s(x_1 - x_3) - y_1 = 4(2 - 0) - 4 \equiv 4 \Mod{7}\\
    \text{O ponto é: }(0, 4)
  \end{align*}
\end{equation}

Os pontos são: $\{(0, 3), (0, 4), (2, 3), (2, 4), (4, 1), (4, 6), (5, 3), (5, 4)\}$

\subsection{Parte 2}

Incluindo o elemento neutro, a ordem do grupo é 9.

\subsection{Parte 3}

O exercício anterior foi construído tomando o ponto $(0, 3)$ como primitivo (ou gerador). Logo, sim,
ele é um elemento primitivo e sua ordem é a mesma do grupo: $ord(\alpha) = \#E = 9$

%\begin{figure}[!htb]
%\includegraphics[width=\textwidth]{fig1}
%\caption{Curva de valores singulares.}
%\label{fig:Curva}
%\end{figure}

\end{document}